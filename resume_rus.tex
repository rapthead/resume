\documentclass[11pt,a4paper,sans]{moderncv}

\moderncvstyle{classic}
\moderncvcolor{blue}

\usepackage{datenumber,fp}
\usepackage[utf8]{inputenc}
\usepackage[russian]{babel}

\usepackage[scale=0.75]{geometry}
\setlength{\hintscolumnwidth}{2cm}
%\setlength{\makecvtitlenamewidth}{10cm}

%----------------------------------------------------------------------------------------
% \newcounter{dateone}%
% \newcounter{datetwo}%
% \setmydatenumber{dateone}{1990}{01}{01}%
% \setmydatenumber{datetwo}{\the\year}{\the\month}{\the\day}%
% \FPsub\result{\thedatetwo}{\thedateone}
% \FPdiv\myage{\result}{365.2425}
% \myage
%----------------------------------------------------------------------------------------
\firstname{Павел}
\familyname{Грибанов}

\title{Инженер-программист}
\extrainfo{
    30 Лет
    \makenewline
    Не женат, детей нет
}
\address{Челябинск}{}
\mobile{(963) 472 58 36}
\email{rapthead@gmail.com}
\homepage{https://github.com/rapthead}
\photo[70pt][0.4pt]{pictures/picture} % The first bracket is the picture height, the second is the thickness of the frame around the picture (0pt for no frame)

%----------------------------------------------------------------------------------------

\AtBeginDocument{
    \hypersetup{colorlinks,urlcolor=color1}
}
\begin{document}

\makecvtitle

%----------------------------------------------------------------------------------------
\section{Образование}

\cventry{2006 -- 2011}{Электронная коммерция, Прикладная информатика (по областям)}{Филиал Московской финансово-юридической академии}{Орск}{\textit{Средний балл -- 4.6}}{}
%\cventry {years}{degree/job title}{institution/employer}{localization}{grade}{description}

%----------------------------------------------------------------------------------------
\section{Опыт}
\subsection{Профессиональная деятельность}
\cventry{\mbox{Июнь 2017} – \mbox{наст. время}}{Инженер-программист}{}{Челябинск}{}{
Backend разработчик node.js/typescript.
Проектирование и разработка сервиса деловой коммуникации. Проект не опубликован, находится в стадии разработки.
Сервис представляет собой RIA совместной работы в режиме реального времени.
\newline{}
\newline{}
\begin{itemize}
    \item разработка бэкенда (nodejs, graphql, aws, heroku, микросервисы, postgresql, redis, infrastructure as a code, CI, TDD, IoC);
    \item code review, консультация фронтенд-разработчиков;
    \item работа по методологиям Agile/Scrum;
\end{itemize}
}

\cventry{\mbox{Июнь 2016} – \mbox{Июнь 2017}}{Инженер-программист}{}{Челябинск}{}{
Fullstack разработчик React/Redux + Python/Flask.
Проектирование и разработка сервиса деловой коммуникации. Проект не опубликован, находится в стадии разработки.
Сервис представляет собой RIA совместной работы в режиме реального времени.
}

\cventry{\mbox{Ноя. 2015} – \mbox{Май 2016}}{Старший программист}{\httplink[PRодвижение]{http://www.web-prural.ru/}}{Челябинск}{}{
Разработка web-сайтов и сервисов (full-stack)
\newline{}
\newline{}
Разработка и поддержка внутренних сервисов и проектов организации (python/django/backbone).
Сервисы seo аналитики, портал лидогенерации.}

\cventry{\mbox{Янв. 2013} – \mbox{Нояб. 2015}}{Инженер-программист}{\httplink[ВЦ СофтСервис]{http://www.softservis.ru/}}{Челябинск}{}{
Full stack разработка web-сайтов (bitrix/php)
\newline{}
\newline{}
Основные достижения:
\begin{itemize}
    \item Наиболее крупные проекты:
    \begin{itemize}
        \item \httplink{rossimvol.ru} - Разработка интернет-магазина для розничных покупателей и дилерской площадки
        \item \httplink{ilovemum.ru} - Редизайн, реализация адаптивного представления, рефакторинг интернет-магазина производителя одежды 
        \item Участие в разработке партнерской площадки производителя шин “Кама”, поддержка legacy-кода
    \end{itemize}
    \item Поддержка корпоративного портала организации
    \item Администрирование web-серверов
\end{itemize}}

\cventry{\mbox{Мар. 2011} – \mbox{Дек. 2011}}{Главный специалист сектора автоматизации}{управлении социальной защиты населения администрации города Орска}{Орск}{}{
    Поддержка сети организации и парка компьютеров.}
%------------------------------------------------

\subsection{Другая занятость}
\cventry{\mbox{Дек. 2011} – \mbox{Дек. 2012}}{служба в Вооруженных Силах Российской Федерации}{}{}{}{}

%----------------------------------------------------------------------------------------
\section{Профессиональные навыки}
\subsection{Языки программирования}
\cvitem{}{
    \cvdoubleitem{JavaScript}{отлично}{Python}{хорошо}
    \cvdoubleitem{PHP}{хорошо}{JAVA}{средне}
    \cvitem{C++}{средне}
    %\begin{itemize}
    %    \item[Perl] отличное знание
    %    \item[PHP] хорошее знание
    %    \item[JavaScript] средний уровень
    %    \item[Python] средний уровень
    %    \item[C++] среднее теоретическое знание
    %\end{itemize}
}

\subsection{JS}
\cvitem{}{
    %\cvdoubleitem{Верстка}{HTML, CSS}{Препроцессоры}{Less, Sass}
    %\cvdoubleitem{Визуальные фреймворки}{Twitter Bootstrap, Zurb Foundation}{Системы сборки}{Grunt, Gulp}
    %\cvitem{Библиотеки}{JQuery, AngularJs, Require JS}
    \begin{itemize}
        \item React/Redux, Backbone.js
        \item ES6, typescript
        \item mocha, chai
        \item Webpack
    \end{itemize}
}

\subsection{DevOps}
\cvitem{}{
    \begin{itemize}
        \item CI/CD
        \item AWS
        \item Infrastructure as Code
        \item Microservices
        \item Linux administration
    \end{itemize}
}

% \subsection{Другое}
% \cvitem{}{
%     \begin{itemize}
%         %\item Глубокое знание и большой опыт работы CMS Битрикс (обмен с 1с, композитный сайт, знание API и методов разработки модулей)
%         \item Глубокое знание ОС GNU/Linux (администрирование, использование, разработка, bash-скриптинг) - опыт более 8 лет
%         \item Умение разбираться в чужом коде
%         \item Понимание и следование парадигме ООП
%         \item Знание систем контроля версий git, mercurial
%         \item Опыт использования системы управления задачами redmine
%         \item Чту pep8, умею автоматический деплой и юнит-тесты
%     \end{itemize}
% }

%----------------------------------------------------------------------------------------
\section{Языки}

\cvitemwithcomment{Русский}{Родной}{}
\cvitemwithcomment{English}{intermediate}{}

%----------------------------------------------------------------------------------------
% \section{Интересы}
% 
% \renewcommand{\listitemsymbol}{-~}
% 
% \cvlistdoubleitem{Горные лыжи}{Компьютерное \textquotedblleftжелезо\textquotedblright}
% \cvlistdoubleitem{Музыка}{Gnu/Linux}
% \cvlistdoubleitem{Free Software}{}

\end{document}
